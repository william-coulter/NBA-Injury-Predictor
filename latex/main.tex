%%%%%%%%%%%%%%%%%%%%%%%%%%%%%%%%%%%%%%%%%
% Lachaise Assignment
% LaTeX Template
% Version 1.0 (26/6/2018)
%
% This template originates from:
% http://www.LaTeXTemplates.com
%
% Authors:
% Marion Lachaise & François Févotte
% Vel (vel@LaTeXTemplates.com)
%
% License:
% CC BY-NC-SA 3.0 (http://creativecommons.org/licenses/by-nc-sa/3.0/)
%
%%%%%%%%%%%%%%%%%%%%%%%%%%%%%%%%%%%%%%%%%

%----------------------------------------------------------------------------------------
%	PACKAGES AND OTHER DOCUMENT CONFIGURATIONS
%----------------------------------------------------------------------------------------

\documentclass{article}

\input{structure.tex} % Include the file specifying the document structure and custom commands

%----------------------------------------------------------------------------------------
%	ASSIGNMENT INFORMATION
%----------------------------------------------------------------------------------------

\title{COMP9417: Machine Learning Project} % Title of the assignment

\author{z5113817} % Author name and email address

\date{University of New South Wales --- \today} % University, school and/or department name(s) and a date

\newcommand{\injurydef}{a physical injury that leaves a player on the injury list for more than 34 days}
%----------------------------------------------------------------------------------------

\begin{document}

\maketitle % Print the title

%----------------------------------------------------------------------------------------
%	Main Contents
%----------------------------------------------------------------------------------------

% TODO: All code for this project is available \href{https://github.com/william-coulter/NBA-Injury-Predictor}{here}.

\section*{Motivation}

In the 1984 NBA draft, Sam Bowie was drafted as the number 2 pick to the Trail Blazers.
It might be shocking to hear that Bowie was drafted one place \emph{above} the hall of fame superstar Michael Jordan.
This was because the Trail Blazers needed a new superstar "big man" to replace the Center they lossed the season before.
Bowie had an impressive 76 games with the Trail Blazers until a fracture in his left tibia put him out for the season.
Even though Bowie followed the recommened recovery time, the rest of Bowie's career was undermined by the recurring injury.
In 10 seasons with the NBA, Bowie only appeared in 511 games.\\

The question is, even though injuries in sport are seen as an unforeseeable tragedy, can a
machine learning model be used to eliminate some of the unpredictability and quantify 
the likelihood that a player will suffer a major injury in the current season?

\section*{The Goal}

Create a model that assigns a likelihood that a player will \emph{suffer a major injury}
given their current performance. Suffering a major injury will be defined \injurydef.

\newpage

% TODO: A background on similar projects / how this project is unique?

\section*{The Data}

\subsection*{Datasets}

The datasets were scraped from various sources such as 
\href{https://www.prosportstransactions.com/basketball}{prosporttransactions} and 
\href{https://www.basketball-reference.com/}{basketball-refrence}. The scrapers
were sourced from \textbf{elap733}'s repository found
\href{https://github.com/elap733/NBA-Injuries-Analysis/tree/master/src/d01_scrapes}{here}.\\

The following datasets can be found in the \textbf{data/raw} directory:
\begin{itemize}
    \item \textbf{player\_stats}: Contains every NBA player's basic statistics for a given season.
    \item \textbf{injury\_list}: Contains information on when players were acquired on and relinquished from NBA injury list.
    \item \textbf{missed\_games}: Contains information on games which players missed (not necessarily due to injury).
    \item \textbf{all\_games\_schedule}: The schedule for every NBA team.
\end{itemize}

The data ranges from 2010 to today.

\subsection*{Cleaning}

For the most part, the data contains pretty clean data with no missing entries
or gibberish values. The datasets were last scraped in 2019 however, so I had to 
do some additional scraping to obtain the latest data. Luckily, the scrapers 
still worked just fine.\\

For cleaning, all that I had to do was append the latest scraped data
with the existing datasets. The cleaned datasets can be viewed at \textbf{data/cleaned}.
The code used to execute this is stored at \textbf{scripts/clean\_data.py}.

\section*{Processing Data}

% This section explains the reasoning and steps take to produce the \textbf{data/processed/physical\_injuries.csv}
% file and what each of the columns mean. The code used to execute this is stored at 
% \textbf{scipts/process\_data.py}.

\subsection*{The Injury List}

This section explains the reasoning and steps take to produce the \textbf{data/processed/physical\_injuries\_2010\_2021.csv}
file and what each of the columns mean. The code used to execute this is stored at 
\textbf{scipts/process\_injury\_list.py}.\\

The injury list in the NBA isn't quite as the name suggests. Rather than being a list
of players who are currently suffering from physical injury, players who miss games
for other reasons can be placed on this list. Other reasons include:

\begin{itemize}
    \item \textbf{illness}: 413
    \item \textbf{surgery}: 253
    \item \textbf{COVID-19}: 11
    \item \textbf{personal reasons}: 11
\end{itemize}

In addition, the list includes players who are \emph{relinquished} from their team 
(put on the injury list) and players who are \emph{acquired} to their team (removed
from the injury list). This means there are essentially 2 entries for 
each individual injury. We want to move the date of this second entry to its own column
so that for each injury we know the dates that player was put on and removed from the 
injury list.\\

Earlier I defined a "major injury" as \injurydef. I defined it in this way
by looking at the distribution of days spent on the injury list in our dataset.

\begin{center}
    \includegraphics[scale=0.6]{day_spent_on_il_distribution.png}    
\end{center}

The red line shows the 80th percentile of this distribution which is at 
\textbf{34} days spent on the injury list
\footnote{Some injuries were marked as "player out for season" which means they don't have another entry in the 
dataset showing their return. For this case, the player was marked down as being on the IL for 100 days. These players
are not included in the histogram however do play a role in this percentile calculation}.
I selected this to be the boundary between a minor and major. Even though this was done somewhat 
arbitarily (I manually decided to use the 80th percentile), I believe this is a
reasonable number since it represents the injuries that left a player on the injury 
list for just over a month and represents only 20\% of all NBA injuries since 2010.\\

The processed injury list dataset is stored at \textbf{data/processed/physical\_injuries\_2010\_2021.csv}.
There are \textbf{8,552} entries in this dataset and \textbf{1,711} are major injuries.

\subsection*{Improving the story}

In his sophomore year, Bowie had suffered a stress fracture in his left tibia. Even though he gave this 

So now we have 1,711 examples where a player were acquired on the injury list
for longer than 34 days. We want to tell a bigger story with these entries by adding some more data
around this.

We want to answer questions such as: How was the player performing before this injury? How intense
was their team's schedule? Had they suffered an injury prior to this?

\subsection*{Correlated fields}



\newpage
\section*{Exploratory Analysis}



\section*{Model Selection}

\section*{Model Evaluation}

\section*{Conclusion}

\subsection*{Nice to haves}

Would've been nice to incorperate the intensity of the schedule some more.

\end{document}
